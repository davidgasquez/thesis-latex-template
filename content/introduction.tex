\chapter{Introduction}
\label{chap:Introduction}

\section{Problem Definition}
\label{sec:Problem Definition}

\subsection{Sources}
\label{sub:Sources}

\section{Motivation}
\label{sec:Motivation}

\section{Objectives}
\label{sec:Objectives}

\section{Scopes}
\label{sec:Scopes}


\section{Demo}
\label{sec:Demo}

Lorem ipsum dolor sit amet, consectetur adipiscing elit. Aliquam augue nisi,
tristique eu faucibus eget, accumsan id est. Vestibulum quis ligula porttitor,
dictum diam maximus, rhoncus quam. Sed ultrices dictum pretium. Morbi ut dui in
enim dictum pretium ut vel nibh. Pellentesque tempor rhoncus ex, vulputate
feugiat tellus molestie sit amet. In porttitor, nibh et scelerisque pharetra,
ligula ligula tristique leo, sed porta arcu erat a sapien. Aenean eget sagittis
lorem.

\begin{table}[h!]
    \centering
    \begin{tabular}{|c c c c|}
        \hline
        Col1 & Col2 & Col2 & Col3 \\
        \hline
        1 & 6 & 87837 & 787 \\
        2 & 7 & 78 & 5415 \\
        3 & 545 & 778 & 7507 \\
        4 & 545 & 18744 & 7560 \\
        5 & 88 & 788 & 6344 \\
        \hline
    \end{tabular}
    \caption{Table to test captions and labels}
    \label{table}
\end{table}

\begin{figure}[H]
    \centering
    \includegraphics[width=3cm]{images/university_logo.png}
    \caption{Logo}
    \label{logo}
\end{figure}

\begin{equation}
    \label{simple_equation}
    \alpha = \sqrt{ \beta }
\end{equation}

\begin{minted}[baselinestretch=1.2]{python}
s = set(['s', 'p', 'a', 'm'])
l = ['s', 'p', 'a', 'm']

def lookup_set(s):
    return 's' in s

def lookup_list(l):
    return 's' in l
\end{minted}
